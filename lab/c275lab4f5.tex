%
\section{\Large{Introducing GNU/Linux}}

The main objective of this lab is to explore the various GNU/Linux distributions and how it has become the center of focus on many major companies to consider Linux as their primary goal of supported products. We also explore some open-source products related to Linux. 

\paragraph{Required Materials: } Any computer with access to the Internet. 
\paragraph{Lab Preparation: } Any computer equipped with Internet web browser. 
\paragraph{Activity Background: } The necessary background for this lab is Chapter 1 and Appendix A in {\bf{A First Course in GNU/Linux}}, Paul Nong-Laolam, Winter 2018 Edition. 

\subsection*{Lab 1.0: Getting to Know the Linux Kernel}

\begin{enumerate}
\item Read Chapter 1 in our textbook, and review Table 1.1.
\item Open a web browser and to {\url{http://www.kernel.org}}. Study the kernel information on this site. 
\item Write the current stable kernel version number and compare it to Table 1.1 in our textbook. 

\vspace{1in}

\item Study the long-term kernel and how many patches or bug fixes have undergone through them. List some of them here:

\vspace{1in}

\end{enumerate}

\subsection*{Lab 1.1: Free GNU/Linux Distributions} 

\begin{enumerate}
\item Open a web browser and navigate to {\url{www.distrowatch.com}}. 
\item Scroll down the web page to view the Latest Updates and new Releases of free GNU/Linux distributions (in the main page). At the same time, take note of the 100 ranking of GNU/Linux distributions on the right. List the top 10 GNU/Linux distributions:

\vspace{1.5in}

\item Scroll back up all the way to the top of the main page, and click the Major Distributions link at the top. 

\item Scroll down to view those major distributions. List their name in the order they each appear. 

\vspace{1in}
 
\item Read and gather some information about a distribution called FreeBSD\index{Free Berkeley Software Distribution}. It is not really a GNU/Linux distribution, but a true UNIX system, which remains the prominent and serious competitor to GNU/Linux. List some of its history and its features:

\vspace{1in}


\end{enumerate}

\subsection*{Lab 1.2: IBM and Linux; Amazon and Microsoft}
In this section, we learn about the major companies and corporates who invest and support GNU/Linux. 

\begin{enumerate}
\item Open a web browser and navigate to {\url{http://www-03.ibm.com/linux/}}. 
\item Take a quick view of this web page.
\item Scroll down the main page all the way down to {\bf{Linux is on fire}} (at the bottom of the page), and click {\bf{IBM LinuxONE and Open Source Demo}} link to view a short video demo on Linux acting as a server for unstructured data. Discuss the benefits of this type of server at applied to trend of information on the Internet.  

\vspace{0.75in}

\item Here is a little interesting article about Amazon pays Microsoft to use Linux: \\
{\url{http://blogs.computerworld.com/15639/amazon_pays_microsoft_for_linux}} \\

\noindent Visit the GNU Project Web site to read its General Public License: {\url{http://www.gnu.org/licenses/gpl-2.0.html}}. Then discuss, in your opinion, the various ways Amazon and Microsoft deal with the legal issues of copyright or patents. 

\vspace{1in}

\end{enumerate}

\subsection*{Lab 1.3: Linux and Its Web Servers: Apache}
In this lab, we explore some of the most powerful and secure web server of Linux\index{Linus Torvalds}. One of them is called Apache. It is open-source and free. 

\begin{enumerate}
\item Open a web browser and navigate to {\url{http://news.netcraft.com/}}. 
\item At the main page, study the data of the current trend of Linux extracted fro mthe Most Reliable Hosting Company Sites in the current month, pay close attention particularly to the type the operating system platforms, downtime, connection speed. What can you say about Linux in this regard?

\vspace{0.75in}

\item In the main page of netcraft, scroll down the main page to view the graph of Apache and other Web servers used within the last 2 decades. List the percentage of Apache Web server used world-wide compared to MS Microsoft. 

\vspace{0.5in}

\end{enumerate}

\subsection*{Lab 1.4: How Linux and MS Windows Can Be Integrated}

\begin{enumerate}
\item Open a web browser and navigate to {\url{http://www.samba.org/}}.
\item Read the information on the main homepage and list the different platforms that Samba supports. You should discover that when you run Samba on your Linux platform, it becomes a file and print server compatible with MS Windows computers. 

\vspace{1in} 
\end{enumerate}

\subsection*{Lab 1.5: Linux Security}
\begin{enumerate}
\item Open a web browser and navigate to {\url{http://www.linuxsecurity.com/}}.
\item We focus only on Linux security matters. Explore the security holes in software, such as the difference between SSH (secure protocol) and Telnet (insecure one). This lab explores and hopes to provide emphasis to educate you about how to avoid security problems and attacks. Linux has built-in firewall and can be deployed on any machine running under Linux.  Scroll down, and in the left pane, click {\bf{Security Tips}}.

\item Linux has a security feature built in to protect its overall infrastructures, called secure-enhanced Linux, aka, {\bf{SELinux}}. This security system is added to Linux to make it far less vulnerable to attacks or break-ins. It should be noted that nothing is secure; only level of security and difficult-to-break-in exist. In terms of security, SELinux is far superior to MS Windows. Even if Linux is successfully attacked, SELinux has the capabilities to limit the attackers the ability to do {\it{anything}} on the computer. Attacks that are so common on Windows are extremely rare in Linux, even on a Linux platform running without SELinux. The contents and links on this web page may change from time to time. Locate a link under the title {\bf{Security Enhanced Linux}}, click to read and gather information regarding SELinux and NSA, particularly NSA's recommendation of Windows over GNU/Linux. Discuss your results here:

\vspace{1.25in}

{\bf{Note:}} We will return to discuss and reflect on this information after we have gained a better and deeper understanding of Linux security features in this course.  

\item Open a web browser and navigate to {\url{http://www.nsa.gov/research/selinux/code/}}. Read and gather information regarding SELinux as an open-source to provide security.  

\end{enumerate}

\subsection*{Lab 1.6: Exploring Different GNU/Linux Distributions} 

We explore the different GNU/Linux vendors available for free download at {\url{www.distrowatch.com}}. 

\begin{enumerate}
\item Open a web browser and navigate to {\url{www.distrowatch.com}} in the URL address field. 
\item Click on the {\bf{Home Page}} link. Scroll down this main page slightly and observe on the lower right of the page for the list/table under the title {\bf{Page Hit Ranking}}. 
\item Click {\bf{Debian}} on the list to get to its main home page, then study the Debian distribution, focusing on: 

\begin{itemize} 
    \item What Linux distro is Debian based on (or derived from) 
    \item Support architecture
    \item Support category (type of platform--desktop, server, etc)
    \item Status 
\end{itemize} 

\item Click the back button (of the browser) to return to the ranking list/table.

\item Click {\bf{AlmaLinux}} on the list to get to its main home page. Study the Debian distribution, focusing on:

\begin{itemize} 
    \item What Linux distro is AlmaLinux based on (or derived from)
    \item Support architecture
    \item Support category (type of platform--desktop, server, etc)
    \item Status 
\end{itemize} 

\item Click the back button (of the browser) to return to the ranking list/table.

\item Click {\bf{Ubuntu}} on the list to get to its main home page. Study the Debian distribution, focusing on:

\begin{itemize} 
    \item What Linux distro is Ubuntu based on (or derived from)
    \item Support architecture
    \item Support category (type of platform--desktop, server, etc)
    \item Status 
\end{itemize} 

\end{enumerate}
